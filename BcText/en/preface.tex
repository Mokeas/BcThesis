\chapter*{A - Introduction}
\addcontentsline{toc}{chapter}{Introduction}

Charles Babbage, the father of the computer, had the first impulse for the invention of a mechanically calculating system at college, when he was tired of mistakes in a table of logarithms. He suggested constructing a machine powered by steam, which could process a larger number of computations than humans while avoiding making mistakes. The idea of using computers to our advantage is carried even now in a highly-paced competitive technology-driven world. Even in the field of journalism, computer science has advanced and the results are getting quite stunning. In 2020  well-known British newspaper The Guardian published an article titled “A robot wrote this entire article. Are you scared yet, human?” \cite{gpt32020robot}, in which AI language generator GPT-3 explains why its existence is not a threat to the existence of mankind. Ignoring the main point of the article, it is well written and I doubt that anyone would recognize that humans did not write the text. Furthermore, in 2022 GPT-3's abilities to write fluent prosaic text were described equivalent to that of a human by The New York Times \cite{johnson2022ai}. This proves the results of AI in the field of language generating.

In this article we discuss numerous challenges of generating text in general and how to conceptually approach developing a language-producing software. This process of generating text (or some linguistic output as discussed below) is in the field of computational linguistics referred to as Natural Language Generation or NLG. These challenges are explained using various specific examples to illustrate exactly what may cause problems and how to prevent them. Also the aim of this paper is to introduce different techniques on how a NLG system can be organised and approached. Everyone who reads the paper should then be able to construct a solution for a NLG problem and especially be aware of the strengths or weaknesses of the solution.

Furthermore, one specific implementation of NLG is presented along with its analysis. This analysis includes description of overall structure, specification of how individual tasks are approached and finally a discussion of the solution. The task fully specified in Chapter (TODO - A - doplnit číslo kapitoly) was to generate a brief article that summarises what happened in a football match. To avoid ambiguity, throughout the article the word "football" refers to a sport, which American English speakers call soccer. This is a first neat example of detailed aspects you need to be aware of when developing a NLG system - meaning of a word can change with different locations. Therefore, it is very important to know the target audience of the generated language. 



