\chapter{A - Natural Language Generation}

\section{A- What is NLG?}
The intuitive meaning of the Natural Language Generation (NLG) is rather obvious, unlike the definitions that usually vary. I will now present definitions of NLG by two different authors:
\begin{enumerate}
	\item “NLG is the subfield of artificial intelligence and computational linguistics that is concerned with the construction of computer systems that can produce understandable texts in English or other human languages from some underlying non-linguistic representation of information.”\label{win-1}\cite{reiter1997building}
	\item “Natural Language Generation (NLG) is the process by which thought is rendered into language.”\label{win-2}\cite{mcdonald2010natural}

\end{enumerate}

The  definition (\ref{win-2}) is much broader and less reliant on specification of what is output and especially input of such a task, which is here defined as “thought”. The problem of identifying the source has been discussed even earlier in \cite{mcdonald1993issues}. He compared the situation to human conversation, namely when the speaker starts deciding what to say. Then the source can be state of mind, current situation, speaker’s intentions etc. These inputs are bordering on the impossible to classify and represent in a computer. The output is defined simply as language, without any stress on the representation of the language. On the other hand, the definition (\ref{win-1}) defines output as understable text, which implies the form of the result is written and additionally restricts the input to be non-linguistic data. However, in the article authors also mention that their survey is focused on written texts, but the principles could be also applied to generating spoken language, which implies the definition could be extended. Examples shown above, and especially the contrast of specifying what output or input can or can not be, explain why it is extremely hard to define the term NLG precisely. 
Let us consider the problem of summarising a book into a brief description. Given this problem and definitions above, we can see that it satisfies only the second vague definition, since the input is purely linguistic. This kind of problem is referred to as text-to-text generation. An example of such a generation is extracting summary from a dialogue of a customer and customer service department described by \cite{liu2019automatic} or pun generation \cite{ritchie2005computational}.
 
Similarly, the initial problem can be transformed into video-to-text generation by switching a book with a movie. In this scenario the line blurs even more. A movie has two components - sound and the video itself. The video has implicitly some semantic meaning and is surely non-linguistic. Howerever, the sound usually contains spoken language and therefore is linguistic. For instance, a video shows Mark pointing at an apple and the sound is “This is a pear.”. The sound itself implies there is a pear, but the message could be summarised as Mark is lying or Mark is not able to recognize a fruit. Finding a correct summarization would be impossible with one of the components missing, because they both affect the overall message, which makes deciding if video-to-text could be classified as NLG difficult. This logic of reasoning can be applied to other problems that vary in their initial inputs such as generating diagnostic report from image (e.g. roentgen) by \cite{zeng2020generating}, which could be characterised as image-to-text generation.

In this article, NLG is perceived as described by \cite{reiter1997building}, meaning creating text from non-linguistic data. This task is often referred to as data-to-text generation. Examples given throughout the article will fit this definition and the idea in general. Methods and approaches mentioned in this article may presumably be applied to any problem concerned with computational linguistics if it is suitable without a need to classify the problem as NLG.

\section{A - Usage of NLG}
The aim of the NLG is to generate documents, articles, messages, emails, descriptions and other forms of texts in order to either reduce workload or to offer a customer a user-friendly interpretation of data in a given language. Various sources papers offer data-to-text implementations operating with different domains, input data and overall aims of the language (which can differ as well):
\begin{itemize}
	\item summarising data and creating reports 
	\begin{itemize}
		\item summarising statistics from a baseball match \cite{puduppully2022data}
		\item summarising geo-referenced data such as map \cite{thomas2007atlas}
		\item creating textual weather forecasts \cite{sripada2014case}
		\item creating report of student's academic performance \cite{}
	\end{itemize}
	\item creating poetry (e.g. in Finnish by \cite{hamalainen2018harnessing})
	\item producing text to persuade reader something is good or bad \cite{carenini2006generating}
\end{itemize}
The most common usage of NLG is to summarise less readable data to more convenient textual form regardless of the domain (sports, weather, geography, etc.). Even though the output of these two tasks is the same, the reason to apply the NLG system is different - compare textual weather forecasts to reports of a student's academic performance. Weather reports are produced in order to enable the general public to find out information about weather since their ability to interpret meteorological raw data is probably lacking. Same reason appears when summarising baseball statistics and maps. On the other hand, there is no doubt that teachers and professors can correctly interpret grades from student’s studies, but the problem is to do that as quickly as possible as well as obtaining the data in a well-readable layout. Generating textual summary automatically resolves both issues effectively at the same time.

The summaries and reports do vary in one more important aspect - the amount of the information and level of terminology. For instance, when creating a weather report for everybody to read some information such as type of the rain or concrete number of mm will not be mentioned in the output text and the information will be abbreviated to “heavy” rain since this is the information the reader is interested in. Contrastingly, when creating a medical report (e.g. from surgical procedure) the terms should be precise and technical (possibly in latin) and the amount of information left out is close to minimum. To conclude, the target audience and their knowledge of the domain will alter the produced text substantially.

The overall aims of the language diverge significantly. In producing poetry the rhymed text filled with phrases calling forth emotions is used to achieve experience as powerful and captivating as possible when reading it. In summaries the aim is to convey the factual information to the reader in an easy and understandable way. Notice that this should be a significant aspect of choosing an approach for the problem. If the goal of the text is to inform, then text is highly recommended to be as simple as possible. On the other hand, if the aim is to captivate, as it is in newspaper articles, then creating simple dull sentences is not a suitable option. 

This was the first glance into the aspects that are crucial when deciding how to approach a language generating problem. These aspects are mentioned many times across the text since the goal of this article is to present to the reader how to approach the NLG and the correct understanding is one of the key parts to develop a solid NLG system.

Is the NLG system good to use every time? Debate whether the software is worth creating is indeed viable and maybe a little underestimated since the articles usually do not analyze feature elaborately. In the real world this is a question of economic resources. Wages and time saved on the human work that has been replaced with software must outweigh the cost of the creation and maintenance of the software. Easy examples of evidently beneficial usage of NLG system is customer service emails generator. Even a simple inserting name to the start of the email, then stating the product, order number and a little survey of satisfaction with the product is quite a trivial email to generate, does the job and saves tons of company time. Possible example of a non-optimal usage is when text is not the most convenient form of data for the user to comprehend. Charts, tables, schemata, maps or pictures could be considerably easier to transfer the intentional message to the end user as these structures are usually attractive and intuitive\footnote{Combination of visual representation of data and text could also be the best alternative in a certain scenario.}. To give you an example, imagine coloured map of a city and using a red line to highlight the path to the nearest tourist information centre in comparison with text composed of verbal instructions where to go. Obviously, map is a better solution due to its simplicity and visuality. 

