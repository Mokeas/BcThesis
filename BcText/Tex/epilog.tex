\chapter*{Conclusion}
\addcontentsline{toc}{chapter}{Conclusion}
The aim of this thesis was to implement an NLG system that generates football articles from non-linguistic data. Along with the implementation, a brief NLG overview was required in order to have baseline knowledge for creating the intended software. We described the process of developing an NLG system end-to-end with tasks that are needed to take into consideration in order to create a fully functioning solution. We have covered the basics of NLG such as different approaches and their core benefits and drawbacks. Moreover, we have illustrated numerous problems that can arise during the development and that are required to be dealt with beforehand.

In the second section we present an implementation of such a text-generating system that uses slightly outdated (for reasons stated throughout the thesis multiple times), but, on the other hand, accessible methods. The implementation is described in detail. Choices made across the solution are usually explained as well. To inspect the implementation even more, please look into the code itself.

As stated in the end of the previous section, this work may not have that many implications. Although software can be used to present the course of the match in a textual and less-schematic representation, the quality of the outputted text is not on the newspaper level (which was not expected). Another problem with incorporating this project into other systems is the specificity of the input and the usage of linguistic realiser Genja, which is not available to the public. However, pieces in between said segments can be used. What is more, this paper can be utilised by other computer scientists, who lack complex insight into the field of computational linguistics and would like to study the subfield of NLG, since the overview of NLG is written from the perspective of a beginner. Furthermore, the text is filled with easy-to-follow examples and possible obstacles during the development to further illustrate knowledge carried across the paper. 
