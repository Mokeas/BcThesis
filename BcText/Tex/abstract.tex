%%% A template for a simple PDF/A file like a stand-alone abstract of the thesis.

\documentclass[12pt]{report}

\usepackage[a4paper, hmargin=1in, vmargin=1in]{geometry}
\usepackage[a-2u]{pdfx}
\usepackage[utf8]{inputenc}
\usepackage[T1]{fontenc}
\usepackage{lmodern}
\usepackage{textcomp}

\begin{document}

%% Do not forget to edit abstract.xmpdata.

Journalism could become a tedious job as its main concern is to create as many
articles as possible, usually prioritising quantity over quality. Some articles are quite
routine and they need to exist just because most of the population prefers text over raw
data. The idea is to ease this job and generate articles, particularly about football in
Czech language, automatically from non-linguistic data.

This thesis is concerned with analysing implementation of such a linguistic software
and moreover offers a brief overview of a Natural Language Generation (NLG) process.
The major focus of this overview is on benefits and drawbacks of different approaches to
NLG as well as describing NLG tasks and its challenges you need to overcome in order
to produce a similar human language (not only Czech) producing program.

\end{document}
