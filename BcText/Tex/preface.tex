\chapter*{Introduction}
\addcontentsline{toc}{chapter}{Introduction}

Charles Babbage, the father of the computer, had the first impulse for the invention of a mechanically calculating system at college, when he was tired of mistakes in a table of logarithms. He suggested constructing a machine powered by steam, which could process a larger number of computations than humans while avoiding making mistakes. The idea of using computers to our advantage is carried even now in a highly-paced competitive technology-driven world. Journalism is a field, where a linguistic software can be implemented in order to at least partially substitute human labour.

This thesis describes various approaches to Natural Language Generation (NLG), the process of automatic production of text, speech or other linguistic types of output. It introduces typical architectures and data structures used and discusses their strengths and weaknesses. We use concrete examples to illustrate some common challenges encountered when building an NLG system.

Furthermore, one specific implementation of NLG is presented along with its analysis. This analysis includes description of overall structure, specification of how individual tasks are approached and finally a discussion of the solution. The task fully specified in \autoref{chap:implementation} was to generate a brief article that summarises what happened in a football match. To avoid ambiguity, throughout the article the word ``football" refers to a sport, which American English speakers call soccer. This is a first neat example of detailed aspects you need to be aware of when developing a NLG system, meaning of a word can change with different locations. Therefore, it is very important to know the target audience of the generated text. 



